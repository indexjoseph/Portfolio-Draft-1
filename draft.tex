\documentclass{article}
\usepackage{graphicx} % Required for inserting images
\usepackage{latexsym}
\usepackage[empty]{fullpage}
\usepackage{titlesec}
\usepackage{marvosym}
\usepackage[usenames,dvipsnames]{color}
\usepackage{verbatim}
\usepackage{enumitem}
\usepackage[hidelinks]{hyperref}
\usepackage{fancyhdr}
\usepackage[english]{babel}
\usepackage{tabularx}
\usepackage{amssymb}
\title{Portfolio Draft 1}
\author{Joseph Oladeji}
\date{March 16 2023}

\begin{document}

\large

\maketitle

\section*{1. Direct proof of an if then statement or a statement with for all and there exists.}

1. Prove: For all integers $n$, if $n$ is odd, then \(n^2+7\) is a multiple of 4.
\newline Description: This proof is a direct proof using the definition for odd integers.
\newline \textbf{Steps}: \newline 1. Identify the statement that has to be provided.
\newline 2. Find any applicable algebra or terms from the supplied information.
\newline 3. Show how the premises and provided information lead to the conclusion using logical reasoning.
\newline 4. Make sure the rationale is sound and that every step is supported.
Explain the conclusion in detail.
\newline \textbf{Proof}: Let \(n\) \in \mathbb{Z} \(\) and be odd.
\newline Then n = 2t + 1, for some integer, t.
\newline Then \(n^2\) + 7 = \((2t+1)^2\) + \(7\) = \(4t^2 + 4t + 8\) = \(4t^2+4t+8\) = \(4(t^2+t+2)\).
\newline Since \(t^2+t+2\) is an integer, \(n^2+7\) is a multiple of 4.
\newline
\newline
2. Prove or disprove:  For every nonzero real number x, there is a real number y such that 3xy=6.
\newline Description: This proof is a direct proof. using the definition for real numbers.
\newline \textbf{Steps}:
\newline 1. Identify the statement that has to be provided.
\newline 2. Find any applicable algebra or terms from the supplied information.
\newline 3. Show how the premises and provided information lead to the conclusion using logical reasoning.
\newline 4. Make sure the rationale is sound and that every step is supported.
Explain the conclusion in detail.
\newline \textbf{Proof}: Let \(x\) \in \mathbb{Z} \(\) and \(x \neq 0\).
\newline Let \(y = 6/3x = 2/x\) So \(y\) is real.
\newline Then \(3xy = 3x(\frac{2}{x}) = 6(\frac{x}{x})= 6\).
\newline So \forall \(x\in \mathbb{R}\) \exists \(y\) where \(3xy = 6\).


\section*{2. Direct proof where the outcome has either an AND or an OR.}
1. Prove: If \(n\) is odd, then \(n^2 + n\) is even and \(( - 1)^2\) is a multiple of 4.
\newline \textbf{Steps}: \newline 1. Identify the statement that has to be provided.
\newline 2. Find any applicable algebra or terms from the supplied information.
\newline 3. Show how the premises and provided information lead to the conclusion using logical reasoning.
\newline 4. Make sure the rationale is sound and that every step is supported.
Explain the conclusion in detail.
\newline Description: This proof is a direct proof using the definition for odd/even integers and contains an AND statement.
\textbf{Proof}: Let \(n\) be odd.
\newline
Then \(n = 2k + 1\) for \(k \in \mathbb{Z}\).
\newline
Then \(n^2 + n = (2k + 1)^2 + 2k + 1 = 4k^2 + 6k + 2\)
\newline
Since \(2k^2 + 3k + 1\) is an integer, n is even.
Also, \((n - 1)^2 = (2k + 1 - 1)^2 = (2k)^2 = 4(k^2)\).
Since \(k^2\) is an integer, n is a multiple of 4.
\newline
\newline
2. Prove: If \(n\) is a multiple of 6, then \(n\) is either a multiple of 4 or  \(n - 2\) is a multiple of 4.
\newline \textbf{Steps}: \newline 1. Identify the statement that has to be provided.
\newline 2. Find any applicable algebra or terms from the supplied information.
\newline 3. Show how the premises and provided information lead to the conclusion using logical reasoning.
\newline 4. Make sure the rationale is sound and that every step is supported.
Explain the conclusion in detail.
\newline Description: This proof is a direct proof using the definition for odd/even integers and contains an OR statement.
\newline
\textbf{Proof}: Let n be a multiple of 6.
\newline
Then n = 6(t) for some integer, t.
\newline
Case 1: t is even
\newline
Then t = 2s, for some integer s.
\newline
So n = 6(2s) = 12s = 4(3s)
\newline
Since s \(\in\) \mathbb{Z}, n is a multiple of 4.
\newline
Case 2: t is odd
\newline
Then n = 2s + 1, for some integer s.
\newline
Then n = 6(2s + 1) = 12s + 6.
\newline
Then n - 2 = 12s + 6 - 2 = 12s + 4 = 4(3s + 1)
\newline
Since 3s + 1 \(\in\) \mathbb{Z}, n is a multiple of 4.
\newline
In all cases, where n is a multiple of 6,
either n is a multiple of 4 or n - 2 is a multiple of 4.
\section*{3. Proof by contrapositive or contradiction.}
1. Proof using contrapositive: If 2n + 5m is a multiple of 3, then \(n\) is not a multiple of 3 or m is not a multiple of 3.

\newline \textbf{Steps}: \newline 1. Identify the contrapositive of the original statement
\newline 2. Find any applicable algebra or terms from the supplied information.
\newline 3. Show how the premises and provided information lead to the conclusion using logical reasoning.
\newline 4. State that by the contradiction the original statement is true or false
\newline Description: This proof is by contrapositive, which is what usually use if we cannot easily define the 'then' part of the statement.
\newline
\textbf{Proof}: Let m be a multiple of 3 and n  be a multiple and n  be a multiple of 3.
\newline
Then m = 3s and n = 3x, where s, x \(\in\) \mathbb{Z}.
\newline
So 2m + 5m = 2(3x) + 5(3s) = 6x + 15s = 3(3x + 5s).
\newline
Since \(3x + 5s\) \(\in\) \mathbb{Z}, \(2n +5m\) is a multiple of 3.
\newline
By contrapositive the statement is true.
\newline
\newline
2. Prove that \(4x^4 - 5 = 10x^3\) has no integer solutions.
\newline \textbf{Steps}: \newline 1. Identify the contradiction (inverse) of the original statement
\newline 2. Find any applicable algebra or terms from the supplied information.
\newline 3. Show how the premises and provided information lead to the conclusion using logical reasoning.
\newline 4. State that by the contradiction the original statement is false
\newline Description: This proof is by contradiction, which is what usually use if we need to assume the opposite of the statement so we can attempt to prove the proof.
\newline
\textbf{Proof}: Let m be a multiple of 3 and n  be a multiple and n  be a multiple of 3.
\newline
Assume that \(4x^6 - 5 = 10x^3\) has an integer solution, call it n. 
\newline
Then \(4n^6 - 10n^3 - 5 = 0\). 
\newline
So \(4n^6 - 10n^3 = 5\), \(2(2n^2 - 5n^3 = 5\).
\newline
Then \(2n^6 - 5n^3 = \frac{5}{2} \notin \mathbb{Z}\).
\newline
But \(2n^6 + 5n^3 \notin \mathbb{Z}\).
\newline
This is a contradiction, so \(4x^6 - 5 = 10x^3\) has no integer solutions.
\section*{4. Proof that one set is a subset of another or NOT a subset of another.}
Let A = \(\{n \in \mathbb{Z} : n = 10t - 5\) for some \(t \in \mathbb{Z}\}\) and Let B = \(\{n \in \mathbb{Z} : n = 5t\) for some \(t \in \mathbb{Z}\}\)
\newline
\newline
 Proof 1. Is A a subset of B? 
\newline \textbf{Steps}: \newline 1. Identify the statement that has to be provided.
\newline 2. Find any applicable algebra or terms from the supplied information. Show x is an element of the first set.
\newline 3. Show how the premises and provided information lead to the conclusion using logical reasoning.
\newline 4. Make sure the rationale is sound and that every step is supported.
Explain the conclusion in detail.
\newline Description: This direct proof with subsets where we're trying to show one set is a subset of another.
\newline
\textbf{Proof}: Let \( x \in A.\)
\newline
Then \(x = 10t - 5,  where t \in 
\mathbb{Z}\).
\newline
Then \(10t - 5 = 5(2t - 1)\).
\newline
Since \(2t - 1 \in \mathbb{Z}\), \(x \in \mathbb{Z}\)
\newline
Thus \(A \subseteq B\)
\newline
\newline
Proof 2. Is B a subset of A? 
\newline \textbf{Steps}: \newline 1. Identify the statement that has to be provided.
\newline 2. Find any applicable algebra or terms from the supplied information. Show x is an element of the first set.
\newline 3. Show how the premises and provided information lead to the conclusion using logical reasoning.
\newline 4. Make sure the rationale is sound and that every step is supported.
Explain the conclusion in detail.
\newline Description: This proof by contradiction with subsets where we're trying to show one set is a subset of another.
\newline
\newline
\textbf{Proof}: Let x = 10 = 5(2), which \(10 \in B\).
\newline
Assume \(x \in A\)
\newline
Then x = 10t - 5, where t \( \in \mathbb{Z} \)
\newline
So 5 + (10) = (10t - 5) + 5, (15)/10 = (10t)/10,
\newline
15/10 = 3/2 = \(t \in \mathbb{Z}\)
However \(t \in \mathbb{Z}\) and \(t = 3/2\).
\newline
By contradiction \(B \subseteq C\)
\section*{5. Proof involving collections of sets.}
Proof 1. If \(B \subseteq A\), \(A \subseteq D\), and \(C \subseteq D\), then \(B \cup C\) \(\subseteq D\).
\newline \textbf{Steps}: \newline 1. Identify the statement that has to be provided.
\newline 2. Find any applicable algebra or terms from the supplied information. Show x is an element of the first sets.
\newline 3. Show how the premises and provided information lead to the conclusion using logical reasoning.
\newline 4. Make sure the rationale is sound and that every step is supported.
Explain the conclusion in detail.
\newline Description: This proof with collections of a sets where we're trying to show a union of two sets is a subset of another.
\newline
\textbf{Proof}: Assume \(B \subseteq A, A \subseteq D\), and \(C \subseteq D\).
\newline
Let \(x \in \) \(B \cup C\).
\newline
Then \(x \in B or x \in C\).
\newline
Case 1: \(x \in B\)
\newline
Since \(x \in B\) and \(B \subseteq A and A \subseteq D\), \(x \in D\).
\newline
Case 2: \( x \in C\)
\newline
Since \(x \in C\) and \(C \subseteq D\), \(x \in D\).
\newline
In all cases where \(x \in B \cup C\), \(x \in D\). Thus \(B \cup C \subseteq D\).
\newline
\newline
Proof 2. If \(B \cap A \subseteq C, and A \cup C \subseteq D, then B \cup C \subseteq D\).
\newline \textbf{Steps}: \newline 1. Identify the statement that has to be provided.
\newline 2. Find any applicable algebra or terms from the supplied information. Show x is an element of the first sets.
\newline 3. Show how the premises and provided information lead to the conclusion using logical reasoning.
\newline 4. Make sure the rationale is sound and that every step is supported.
Explain the conclusion in detail.
\newline Description: This proof with collections of a sets where we're trying to show a intersect of two sets is a subset of another.
\newline
\textbf{Disproof}: Let \(A = \{1, 2 ,3\}\),\(B = \{2, 3\}\), \(C = \{2\}\), and \(D = \{1, 2, 4, 5\}\)
\newline
Note that $A \cap B \subseteq C$ and $A \cup C \susbseteq D$.
\newline
However $B \cup C$ = $\{2, 3\}$ which is not a subset of $D$, because $3 \in B \cup C$ and $3 \notin D$.
\newline
Therefore the original statement is false.

\end{document}
